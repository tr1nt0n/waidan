\documentclass[12pt]{article}
\usepackage{fontspec}
\usepackage[utf8]{inputenc}
\setmainfont{Bodoni 72 Book}
\usepackage[paperwidth=11in,paperheight=17in,margin=1in,headheight=0.0in,footskip=0.5in,includehead,includefoot,portrait]{geometry}
\usepackage[absolute]{textpos}
\TPGrid[0.5in, 0.25in]{23}{24}
\parindent=0pt
\parskip=12pt
\usepackage{nopageno}
\usepackage{graphicx}
\graphicspath{ {./images/} }
\usepackage{amsmath}
\usepackage{tikz}
\newcommand*\circled[1]{\tikz[baseline=(char.base)]{
            \node[shape=circle,draw,inner sep=1pt] (char) {#1};}}

\begin{document}

\begingroup
\begin{center}
\end{center}
\endgroup

\vspace{11\baselineskip}

\begingroup
\begin{center}
\huge P R O G R A M \hspace{4mm} N O T E
\end{center}
\endgroup

\vspace{3\baselineskip}

\begingroup
\hspace{4mm} {\setmainfont{Source Han Serif SC}\selectfont 外丹} ( transl. ``external alchemy" ) is a Daoist practice involving the mixture of minerals, primarily red cinnabar, into an elixir which, when consumed, bestows immortality upon the drinker. This piece relates to that practice through the process of gathering and mixing with a transcendent intention. The stones cast and touched by the musicians, the sonic ideas gathered as independent entities and assembled as forms, and the identities and experiences of the artists involved are imagined as the minerals to be brought together, drunk, and metabolised as eternity. Additionally, the piece relates more broadly to ritual itself. 

\endgroup
\begingroup
\hspace{4mm} The musical ritual has been a pancultural phenomenon for as long as anything like religion has existed in humans. In fact, there is reason to believe that the musical ritual even predates religion, and homo sapiens.
\endgroup

\begingroup
\hspace{4mm} As divinity develops, we observe, most obviously within pantheons, gods utilised to assign ego to non-objectified passions. The passions of gratitude for security provided by our natural surroundings become a love for the anthropomorphised environment, Gaia, Mother Earth. The passions of wrath, justice, and catharsis become a reverence to and invocation of the power of war and storm deities such as Ba'al and Yahweh. Perhaps the most essential of all, the passions of admiration, desire, the intimacy of obsession permeate not only the obvious Aphrodites of human culture, but even deities whose purposes focus not solely on the objectification of infatuation and worship. Worship becomes a perpetuation of itself, directed toward itself. 
\endgroup

\begingroup
\hspace{4mm} However, these and all passions are not objects. They are intangible. I have seen how human beings fear and despise intangible happenings. Hence the attachments of names, faces, identities to these passions which we undoubtedly, viscerally, and torturously invisibly experience. 
\endgroup

\begingroup
\hspace{4mm} The only other unquestionably real but nevertheless intangible, non-objective happening which I have experienced so far is instrumental music. And this happening most notably originates, unlike the passions, from outside of the individual's central nervous system. Yes, it is intangible, but nevertheless deeply sensual. 
\endgroup

\begingroup
\hspace{4mm} This piece draws its rituals from the belief that the musical ritual is not reactant to, rather creative of divinity and transcendence. The somatic gestures of the interpreters, the organisation of sitting and standing, the spatialisation of musicians on the stage and electronics off the stage, the negotiation of repetitions, of the shifting balance between listening and watching, all serve not as invocations of exterior symbols, rather conjurations of the as-yet nonexistent, and the perpetually invisible. 
\endgroup

\begingroup
\hspace{4mm} This is a piece about immortality, about magic, and about love. Over all, it presents the three as an inseparable unity.
\endgroup

\end{document}