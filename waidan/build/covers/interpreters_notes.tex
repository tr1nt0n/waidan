\documentclass[12pt]{article}
\usepackage{fontspec}
\usepackage[utf8]{inputenc}
\setmainfont{Bodoni 72 Book}
\usepackage[paperwidth=11in,paperheight=17in,margin=1in,headheight=0.0in,footskip=0.5in,includehead,includefoot,portrait]{geometry}
\usepackage[absolute]{textpos}
\TPGrid[0.5in, 0.25in]{23}{24}
\parindent=0pt
\parskip=12pt
\usepackage{nopageno}
\usepackage{graphicx}
\graphicspath{ {./images/} }
\usepackage{amsmath}
\usepackage{hyperref}
\usepackage{tikz}
\newcommand*\circled[1]{\tikz[baseline=(char.base)]{
            \node[shape=circle,draw,inner sep=1pt] (char) {#1};}}

\begin{document}

\vspace*{1\baselineskip}

\begingroup
\begin{center}
\huge NOTES FOR THE INTERPRETERS
\end{center}
\endgroup

% things to explain: 
% piano clusters black key, white key, black and white key

\begingroup
\textbf{Staging:} The position of the instruments on stage is illustrated below, with arrows emanating from the names of the interpreters indicating which direction they are facing. Please note that the musicians should orient themselves based on \textbf{cardinal directions}, regardless of the orientation of the concert space. 
\begin{center}
\includegraphics[scale=0.4]{staging.png}\\
\end{center}
The violist is enclosed by four curtains or walls. The two curtains/walls facing \textbf{east} must be able to be opened during the performance, referred in the score as the viola's \textbf{``gate."}
\endgroup

\begingroup
\textbf{General: \circled{1} This score is collaboratively interpreted}, meaning that the timing and syncronisation of events relies heavily on the interpreters' awareness of each other's actions. Therefore, individual parts are not provided for this work. \textbf{\circled{2}} After temporary \textbf{accidentals}, cancellation marks are printed also in the following measure ( for notes in the same octave ) and, in the same measure, for notes in other octaves, but they are printed again if the same note appears later in the same measure, except if the note is immediately repeated. \textbf{\circled{3} Microtones} present in this score are \textbf{quarter-tones}, \textbf{third-tones}, and \textbf{rational intervals}. \\
\circled{1.} \includegraphics[scale=0.05]{quarter_flat.png} indicates a \textbf{quarter-tone flat}. \\
\circled{2.} \includegraphics[scale=0.05]{quarter_sharp.png} indicates a \textbf{quarter-tone sharp}.\\
\circled{3.} \includegraphics[scale=0.05]{third_tone.png} indicates a \textbf{third-tone}, with an arrow above or below indicating whether the pitch is raised or lowered. In this example, the pitch should be played \textbf{two third-tones flat}. \\ 
\circled{4.} \textbf{Justly tuned intervals} are indicated by the use of \textbf{Helmholtz-Ellis accidental system} combined with \textbf{cent deviations from equal temperament} for use with an electronic tuner. When no example pitch is given with the cent deviation, the mark is a deviation of the nearest ``standard” accidental. In the absence of electronic tuners, approximations of these deviations are acceptable. When Helmholtz-Ellis notation is not given, the pitches are to be played as usual. \\
\textbf{\circled{4} Various forms of temporal notation} are utilised in this score. When a note head appears on the staff alone, its duration should be freely interpreted based on its spatialisation within its larger rhythmic context. This larger rhythmic context could be the duration of the measure, or, in the case illustrated below, a bracket indicating the duration of the durational space to play within: \\
\begin{center}
\includegraphics[scale=0.4]{duration_bracket.png}\\
\end{center}
\endgroup 

\pagebreak

\begingroup
\textbf{Box notation} is also sometimes used, wherein musicians are given an example figure to improvise with for the duration of the section: \\
\begin{center}
\includegraphics[scale=0.4]{box_notation.png}\\
\end{center}
\textbf{\circled{5} Physical gestures} are sometimes prescribed using \textbf{illustrations}, detailed below: \\
\circled{1.} \includegraphics[scale=0.1]{position_one.png} \\ indicates to form a triangle by touching the thumbs and index fingers of both hands together, extend the remaining fingers, and hold the resultant hand shape in front of the forehead. \\ 
\circled{2.} \includegraphics[scale=0.1]{position_two.png} \\ indicates to form the same triangle shape with the hands described above, and to hold that shape above the head, instead of in front of the forehead. \\ 
\circled{3.} \includegraphics[scale=0.1]{position_three.png} \\ indicates to extend the fingers of both hands, and hold them on either side of the face, palms facing outwards.\\ 
\circled{4.} \includegraphics[scale=0.1]{position_four.png} \\ indicates to place the palms of each hand flat against each other, intersect the ring fingers, and hold the resultant hand shape above the head. \\ 
\circled{5.} \includegraphics[scale=0.1]{position_five.png} \\ indicates to form the same hand shape described above, and to hold that shape in front of the body, rather than above the head. \\ 
\endgroup 

\begingroup
These positions can be \textbf{held}, indicated by a \textbf{dashed, hooked line} spanning the time the position is maintained, or \textbf{transitioned between}, indicated by an \textbf{arrow} between the two positions to be interpolated.
\textbf{\circled{6} Sitting and standing} are prescribed with boxed text above the staff. The entire ensemble is always directed to sit or stand together while playing. 
\textbf{\circled{7} Stones} are given to each musician. The violist plays with \textbf{seven small stones} which are struck against each other, while the other musicians play with \textbf{three small stones} and \textbf{one large stone}, striking the small stones atop the larger, mounted one. The stones may be played in the following ways: When instructed to \textbf{cast} the stones, the musician should hold the stones in their hands and then throw them on the floor, like dice. When instructed to \textbf{strike} the stones, the musician should strike the stones against each other. When instructed to \textbf{scrape} the stones, the musician should drag the stones across each other at a high pressure. When instructed to \textbf{rub} the stones, the musician should drag the stones across each other at medium-light pressure. When instructed to \textbf{crunch} the stones, the musicians should hold the small stones in the hand, and press them tightly against each other while moving them, creating sounds from their friction against each other. When rubbing the stones, sometimes the direction of the motion is prescribed using arrow-shaped note heads, as below: \\
\begin{center}
\includegraphics[scale=0.2]{stone_rub_directions.png}
\end{center}
\textbf{\circled{8} Speeds of tremoli} may be interpolated, indicated by arrows between two tremolo speeds above the staff. A tremolo with \textbf{one line} indicates \textbf{tremolo largo}, \textbf{two lines} indicates \textbf{tremolo moderato}, and \textbf{three lines} indicates \textbf{tremolo stretto}.
\begin{center}
\includegraphics[scale=0.2]{bellow_tremolo.png}
\end{center}
\textbf{\circled{9} Measures 73 - 84} feature rhythms in a \textbf{staff for the conductor} with dynamic contours attached to them. During this passage, the conductor should use all of the bodily resources available to them, such as the shape and movement of the hands, the posture of the spine and legs, the facial expression, to lead the ensemble in a unison performance of the written rhythms. In this passage, dynamics can be understood more as markers of expressive intensity than as actual volumes. 
\endgroup 

\begingroup
\textbf{Electronics: \circled{1} Four tape files} are provided with this score, labelled, ``\textbf{tape I.wav}," ``\textbf{tape II.wav}," and so on. The start-time of each file in the piece is indicated with text above the staff. The contents of the tape need not be perfectly syncronised to the actions of the musicians. \textbf{\circled{2} Two toggle-effects}, a granulation effect and an envelope-following effect, are provided in a Supercollider file. These effects process the signals of each individual instrument live. The on-and-off toggling of the effects are signaled using text indications in the staff, shortening the granulator effect as ``\textbf{Gran}," and the envelope-following effect as ``\textbf{Env}," followed by an OFF or ON indicator. \textbf{\circled{3} Each instrument is amplified} using at least one dynamic microphone. \textbf{\circled{4} Three loudspeakers} are positioned in an arch on the left side of the audience. \\
\textbf{The viola} is sent to the \textbf{center speaker}. \\
\textbf{The saxophone} is sent to the \textbf{center speaker}. \\
\textbf{The cello} is sent to the \textbf{left speaker}. \\
\textbf{The harp} is sent to the \textbf{right speaker}. \\
\textbf{The accordion} is sent to the \textbf{right speaker}. \\
\textbf{The piano} is sent to the \textbf{left speaker}. \\
\endgroup

\begingroup
\textbf{Strings: \circled{1} Two tablature clefs} are used. \\ 
\endgroup

\begingroup
\circled{1.} \includegraphics[scale=0.1]{viola_string_clef.png} \\ indicates the bow's contact point on \textbf{strings II and III}. The first line is \textbf{behind the bridge}, the second line is \textbf{on the bridge}, the third line is \textbf{above the edge of the fingerboard}, and the bottom line is \textbf{at the top of the neck} near the nut. \\
\circled{2.} \includegraphics[scale=0.1]{four_line_staff.png} \\ indicates to play on open strings. The top line is \textbf{string I}, the next is \textbf{string II}, and so on. \\
\endgroup

\begingroup
\textbf{\circled{2} Bow angle} is indicated with \textbf{degrees in arrows above the staff}, wherein pointing the tip of the bow \textbf{towards the nut} is \textbf{0°}, holding the bow \textbf{completely perpendicular to the string} is \textbf{90°}, and pointing the tip of the bow \textbf{towards the tailpiece} is \textbf{180°}. \textbf{\circled{3} Degrees of spazzolato} are also prescribed, wherein full spazzolato indicates only vertical bow motion, and fractional additions such as \textbf{2/3 spazzolato} gradually increase the amount of horizontal bow motion, resulting in a diagonal bowing. \textbf{\circled{4} XFB} is an abbreviation of ``\textbf{extremely fast bowing}," which is a kind of irregular tremolo using extreme flautando with the bow only very lightly skimming the string. \textbf{\circled{4} Crunching the bow hair on the back of the instrument} is indicated with a curved, double-sided arrow underneath the note head, and a text instruction reading ``back of body." When this notation is seen, the bow should be continually twisted on the back of the instrument, causing friction noises between the instrument, the hair of the bow, and the wood of the bow. \\
\begin{center}
\includegraphics[scale=0.2]{back_of_body_pressure.png} \\
\end{center}
The pressure of the bow is indicated graphically in the staff using black filled-in contours. 
\endgroup

\begingroup
\textbf{Viola: \circled{1} The viola is prepared} with \textbf{a piece of Patafix} on string II near the bridge, and on string IV near the nut. Strings II and IV should not be played on unless directed. \textbf{\circled{2} Scordatura}: String III is detuned to an \textbf{E-flat 3} at a ratio of \textbf{7/6} over the pitch of string IV. String I is detuned to an \textbf{E 4} at a \textbf{major second} over string II. This score is written in \textbf{sounding pitch}. 
\endgroup

\begingroup
\textbf{Saxophone: \circled{1}} The saxophonist plays \textbf{two saxophones}, a \textbf{soprano} and a \textbf{baritone}. \textbf{\circled{2}} The soprano saxophone is transposed a \textbf{major second higher} than concert pitch. The baritone saxophone is transposed \textbf{an octave plus a major sixth higher} than concert pitch. \textbf{\circled{3}} The soprano has its mouthpiece removed after measure 17, after which it is played by buzzing the lips into the open tubes, with an embouchure as if playing a trumpet. When playing this technique, glissando contours are given for the interpreter to approximate using the relative tightness or looseness of the lips.
\endgroup

\pagebreak

\begingroup
\textbf{\circled{4}} From measure 118, a notation is used which represents breath pressure as a black, filled-in contour: \\
\begin{center}
\includegraphics[scale=0.3]{saxophone_breath_pressure.png} \\
\end{center}
While playing the fingerings provided, the higher peaks should be approximately interpreted as higher breath pressure, with a range between \textbf{barely exhaling} ( lowest ) and \textbf{overblowing} ( highest ). \\
\textbf{\circled{5} Buzzing the teeth on the reed} is coupled with a glissando indicating the pitch contour of the buzzing: \\
\begin{center}
\includegraphics[scale=0.2]{saxophone_teeth_on_reed.png} \\
\end{center}
\endgroup

\begingroup
\textbf{Cello: \circled{1} The cello is prepared} with \textbf{a strip of aluminum foil} woven between the strings behind the bridge. \textbf{\circled{2} Scordatura}: String IV is detuned a minor sixth to \textbf{E 2}. This score is written in \textbf{sounding pitch}. \textbf{\circled{3}} The cellist should be equipped with a \textbf{rough sponge} with which to rub the body of the instrument, above the waist. 
\endgroup

\begingroup
\textbf{Harp: \circled{1} The harp is prepared} with \textbf{newspaper} woven between the centers of the \textbf{F 1}, \textbf{G 1}, and \textbf{A 1} strings. When playing on these strings a two line staff is used wherein the \textbf{top line} indicates to play on the newspaper-damped strings with the hand close to the \textbf{upper frame}, the \textbf{bottom line} indicates to play on the newspaper-damped strings with the hand close to the \textbf{lower frame}, and the \textbf{space between} indicates \textbf{approximate positions between the two}.\\
\begin{center}
\includegraphics[scale=0.2]{harp_newspaper_staff.png} \\
\end{center}
\textbf{\circled{2} A thunder-effect} accomplished by striking a low cluster so forcefully that the strings strike against one another is indicated using Salzedo's notation: \\
\includegraphics[scale=0.1]{salzedo_thunder.png} \\
\textbf{\circled{3}} The harpist should be equipped with \textbf{two bows}.
\endgroup 

\begingroup
\textbf{Accordion \circled{1}} When instructed to play on the \textbf{bellows}, the interpreter should place their hand between the bellows and rapidly move the hand back and forth, striking the side of the instrument with the bellows' movements. \textbf{\circled{2} The speed of trills} are sometimes indicated using the tightness of a trill spanner's curves: \\
\begin{center}
\includegraphics[scale=0.2]{spatial_trills.png} \\
\end{center}
\endgroup

\begingroup
\textbf{Piano \circled{1} The piano is prepared} with \textbf{thin chain} laid across all of the strings, and \textbf{heavily-rosined jewellery-wire} woven between the \textbf{B 3}, \textbf{F sharp 4}, \textbf{C 5}, \textbf{E 5} and \textbf{F 5} strings, which is used to bow these strings. \textbf{\circled{2}} The pianist should be equipped with a \textbf{shorter, thick chain}, and \textbf{two medium yarn mallets}. \textbf{\circled{3} Clusters} are normally to be played as groups of black and white keys within the cluster's range. If, however, the cluster has a \textbf{sharp accidental} above or beneath it, that cluster should contain only \textbf{black keys}, and if it has a \textbf{natural accidental} above or beneath it, that cluster should contain only \textbf{white keys}.
\endgroup

\end{document}